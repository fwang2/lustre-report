\section{MDC and Lustre Metadata}
\label{sec:lustre-mdc}

\subsection{MDC Overview}

The MDC component is layered below Lustre Lite. It defines a set of
metadata related functions that Lustre Lite can call to send metadata requests
to MDS.  The set of functions is implemented in \url{lustre/mdc}, and we 
discuss them in Section \ref{sec:stripingapi}.

Lustre Lite passes the request parameters in a \url{mdc_op_data} data
structure, and all requests must eventually be converted to a structure known
as the \url{ptl_request}.  Therefore, before a RPC request can be carried out,
there are several steps for preparing it (\textit{packing}). Within the
\url{mdc_lib.c}, there are a number of functions defined for this purpose. In
turn, some of these functions actually invoke packing helper methods provided
by the Portal RPC layer.

Once \url{ptl_request} is ready, MDC can then invoke \url{ptlrpc_queue_wait()}
to send the request.  This is a synchronous operation, and all metadata
operations with intent are also synchronous.  There are other ptlrpc methods
for send operations and used for metadata operations without intent
(\textit{mdc\_reint})---this is done in code \url{mdc_reint.c}.

For the most part, the driver of converting from \url{mdc_op_data} to
\url{ptl_request} and the call to enqueue the request are done in function
\url{mdc_enqueue}, implemented in \url{lustre/mdc/mdc_locks.c}.


\subsection{Striping EA}
\label{sec:stripingea}

Depending on where the striping EA is created or used, there are three formats:

\begin{itemize}

  \item On-disk format, used when it is stored on MDS disk, described by
\url{struct lov_mds_md}.

  \item In-memory format, used when it is read into the memory and unpacked,
described by \url{struct lov_stripe_md}.

  \item User format, used when the information is to be presented to the user,
described by \url{struct lov_user_md}.

\end{itemize}


The user format differs from on-disk format in two aspects: 

\begin{itemize}

\item The user format has \url{lmm_stripe_offset}, which on-disk format does
not have. This field is used to transfer the \url{striping_index} parameter to
Lustre when user wants to set a stripe.

\item The user format has \url{lmm_stripe_count} as 16-bit, whereas on-disk
format has this field as 32-bit.

\end{itemize}

%\begin{figure}[htb]
%\includegraphics[width=5in]{img/struct_lov_mds_md}
%\end{figure}


\subsection{Striping API}
\label{sec:stripingapi}

Five types of APIs are defined to handle striping EA. These are listed below.

\begin{description}

  \item[set/get API] To set or get a stripe EA from storage. It operates
  on on-disk striping EA.

  \begin{Verbatim}
  int fsfilt_set_md(struct obd_device *obd, struct inode *inode, 
                 void *handle, void* md, int size, const char *name)
  int fsfilt_get_md(struct obd_device *obd, struct inode *inode, 
                 void *md, int size, const char *name)            
  \end{Verbatim}

  Here, \url{md} is the buffer for the striping EA, \url{handle} is the journal
  handle, and \url{inode} refers to the MDS object.
  
  \item[pack/unpack API] As striping EAs are stored in packed format on disk, they
   are unpacked after \url{fsfilt_get_md()} is called. These APIs can be
   used for both on-disk and in-memory striping EA. 
  
  \begin{Verbatim}
  int obd_packmd(struct obd_export *exp, struct lov_mds_md **disk_tgt,
                 struct lov_stripe_md *mem_src)
  int obd_unpackmd(struct obd_export *exp, struct lov_stripe_md **mem_tgt,
                  struct lov_mds_md *disk_src, int disk_len)
  \end{Verbatim}
  
  Here, \url{mem_src} points to the in-memory structure of striping EA, and \url{disk_tgt}
  points to the disk structure of striping EA. Conversely, \url{disk_src} is the source
  for on-disk striping EA, while \url{mem_tgt} is the  target for in-memory
  striping EA.
  
  \item[alloc/free API] Allocates and frees striping EA in memory and on-disk. 
  \begin{Verbatim}
  void obd_size_diskmd(struct obd_export *exp, struct lov_mds_md *dis_tgt)
  int obd_alloc_diskmd(struct obd_export *exp, struct lov_mds_md **disk_tgt)
  int obd_free_diskmd(struct obd_export *exp, struct lov_mds_md **disk_tgt)
  int obd_alloc_memmd(struct obd_export *exp, struct lov_stripe_md **mem_tgt)
  int obd_free_memmd(struct obd_export *exp, struct lov_stripe_md **mem_tgt)
  \end{Verbatim}
  
  \item[striping location API] Returns data object location information from striping EA.
  
  \begin{Verbatim}
  obd_size love_stripe_size(struct lov_stripe_md *lsm, obd_size ost_size, int stripeno)
  int lov_stripe_offset(struct lov_stripe_md *lsm, obd_off lov_off, 
                     int stripeno, obd_off *obd_off)  
  int lov_stripe_number(struct lov_stripe_md *lsm, obd_off lov_off)
  \end{Verbatim}

  Here, \url{lov_off} is the file logical offset. The \url{stripeno} is the stripe number
  of the data object. 

  \item[lfs API] User-level API to handle striping EA and is used by the \url{lfs}
  utility.

  \begin{Verbatim}
   int llapi_file_get_stripe(const char *path, struct lov_user_md *lum)
   int llapi_file_open(const char *name, int flags, int mode, 
      unsigned long stripe_size, int stripe_offset, int stripe_count,int stripe_pattern)
   \end{Verbatim} 

    Here, \url{llapi_file_get_stripe()} will return a user striping EA given a path, and
    \url{llapi_file_open()} opens/creates a file with a user specified stripe
    pattern. It is also worth noting that \url{stripe_offset} is the same as
    \url{stripe_index} used in user space, and it is the OST index of the first
    stripe.
    
\end{description}

