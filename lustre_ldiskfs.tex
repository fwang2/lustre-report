\section{Lustre Disk Filesystem: ldiskfs}

The ldiskfs (also sometimes wrongly dubbed the Linux ext4 filesystem) is a
heavily patched version of the Linux ext3 filesystem and is developed and
maintained by Sun Microsystems. ldiskfs is a superset of Linux
ext3 and ext4 filesystems. Currently, it is only used by Lustre on servers
as an underlying local filesystem. This section provides a brief overview
of ldiskfs and its differences from the ext3 filesystem.

The main differences between the ext3 and ldiskfs filesystems in terms of their
respective I/O paths are listed here. 

\begin{itemize}

\item In ldiskfs, allocation/block lookup is done up front with inode lock
held, but then inode lock is dropped while the I/O is being submitted to disk.
This greatly improves I/O concurrency because inode is locked for only a short
time.

\item In ldiskfs, allocation is done for the entire RPC (1MB, or in other
words, 256$\times$4KB blocks) at one time, instead of 4KB block-at-a-time like VFS
does.  This avoids lots of overhead in repeated bitmap searching and also
allows more efficient extent finding.

\item In ldiskfs, writes are currently fully synchronous to the client (no
reply until data and metadata are on disk), but this is changing. The newly
developed ``async commit" patch will reply to client after data is on disk, but
before metadata is committed in any transaction.  Also, the new ROC (1.8) will
put write pages into cache before flushing them to disk, in preparation for a
fully asynchronous write cache on the OSS.

\item In ldiskfs, currently journal flush is forced after every write RPC.

\end{itemize}

In ldiskfs, as with any other underlying local filesystem to be used by
the Lustre servers, the policies listed here can affect the I/O path.

\begin{itemize}

\item \textit{Client-controlled RPCs in flight}, \textit{RPC size} and
\textit{cached dirty data per OST}.  \textit{RPC size} is preferred to be a
multiple of the underlying RAID stripe size to avoid read-modify-write
operations. Also, $RPC\,size \times num\,of\,RPCs\,in\,flight \times
num\,of\,clients$ should take into account the bandwidth of the link and
bandwidth of underlying block devices to avoid big latency and requests piling
up at contention points.

\item \textit{Server-controlled number of I/O threads}. This currently limits the
amount of I/O to backend devices in total at any given time.

\item \textit{Striping policy}. It determines how many OSTs participate in I/O
to a given file.

\end{itemize}

\subsection{Kernel Patches}
\label{ldiskfs-kernel-patches}

The ldiskfs and Lustre require a set of Linux kernel patches that not only adds new
features on top of ext3 but also improves the performance. The following list provides
the essential set of kernel patches for ldiskfs, although some might not be
needed for certain use cases (e.g.~\url{sd_iostats}).

\begin{itemize}

\item \url{jbd-2.6.10-jcberr.patch}: journal callback patch that allows
replying to clients with the~\url{last_committed} when data is committed at
least to the journal. This also means that the data is  recoverable by the
local filesystem without the client support.

\item \url{jbd-stats-2.6-sles10.patch}: provides statistics about transaction size,
time, etc.

\item \url{iopen-misc-2.6.12.patch}: allows ~\url{open-by-inode}
for~\url{getattr-by-fid} and recovery. 

\item \url{dev_read_only-2.6-fc5.patch}: discards writes to block devices for
testing and~\textit{failback} of OST/MDT without blocking on in-flight I/O.

\item \url{sd_iostats-2.6-rhel5.patch}: provides optional SCSI device statistics.  

\item \url{blkdev_tunables-2.6-sles10.patch}: allows submitting larger I/Os
through the block layer.

\item \url{i_filter_data.patch}: allows hooking MDS/OSS data from the in-memory
inode.

\item \url{quota-fix-oops-in-invalidate_dquots.patch}: fixes bug in upstream
kernel quota code. This is fixed in the 2.6.17 kernel.

\item \url{jbd-journal-chksum-2.6-sles10.patch}: provides journal transaction
checksums to detect on-disk corruption.

\end{itemize}

\subsection{Patches: ext3 to ldiskfs}
\label{ext3-ldiskfs-patches}

Besides the patches listed above, ldiskfs requires a number of patches on top 
of the ext3 source base for increased performance and functionality. The following list 
shows the patches required for a ldiskfs filesystem.

\begin{itemize}

\item \url{ext3-wantedi-2.6-rhel4.patch}: allows creating specific inodes by
number and is used for recovery. Suppose the client did an open/create and got
a reply from the server with a specific inode number. Let's assume the server
crashes at this point but the creation is not reflected to disk yet, so the
information is lost. After the server comes back up and enters recovery, the
client connects back and sends its open/create request back again, specifying
the earlier communicated inode number from the server. In Lustre 2.0 this will
not be needed, as there will be a different set of Lustre inode numbers than
the actual respective inode numbers on disk. On servers a functionality for
mapping Lustre inode numbers to actual on disk inode numbers will eliminate the
dependency on this patch. 

\item \url{iopen-2.6-fc5.patch}: allows lookup inodes by number. Typically, one
needs the full path to look up a file, but Lustre does not have this
functionality in its protocol. When one does a lookup, it only has the parent
inode number and child name, and normally this is enough. But if Lustre is in
recovery mode, all previous lookup steps are lost and Lustre needs a method to
find the parent inode number. This patch provides this functionality. This
problem does not occur in ext3 because the filesystem is local, and at the time
of a crash, everything goes down, and at the time of recovery (or reboot), all the
look up functionality starts from the root directory.

\item \url{ext3-map_inode_page-2.6-suse.patch}: allocates data blocks for bulk
read/write operations for later submission for I/O. This patch creates an API 
that can be used to map~\url{inode_page} function in the Lustre~\url{fsfilt} layer.
Lustre needs this patch because it allocates blocks ahead of time to enhance the 
metadata performance. 

\end{itemize}

The next five patches can be grouped together as they provide~\url{extent}
format capability on top of the ext3 filesystem. They are also used in the
Linux ext4 filesystem. The ext3 filesystem address files in blocks, and every
block has a 4 byte record in inode with the block number where the actual data
for this block is stored. For long files, this results in storing lots of block
information in the inode, which is not very efficient. The~\url{extent} concept
defines a range of blocks to be used as a single entity in block allocation for
large files. The maximum for a range of blocks in an~\url{extent} is 128MB.
This is due to the fact that for every 128MB there is a block holding the
bitmap of the next 128MB of allocation. The only exception for this is files
with holes, where a hole, no matter how big it is, can be represented with a
single~\url{extent}. For a very large file in Lustre with the following set of
patches, one will have a list of~\url{extents}, with each~\url{extent} being
128MB less 4KB for the bitmap mapping for that particular 128MB~\url{extent}
group.  The~\url{extent} concepts improves the performance such that, not only
at the time of allocating but, for example, when unlinking a file, one only
needs to provide the block information defining the~\url{extent} for that
particular file.
 
\begin{itemize}

\item \url{ext3-extents-2.6.16-sles10.patch}:

\item \url{ext3-extents-fixes-2.6.9-rhel4.patch}:

\item \url{ext3-extents-multiblock-directio-2.6.9-rhel4.patch}:

\item \url{ext3-extents-search-2.6.9-rhel4.patch}:

\item \url{ext3-extents-sanity-checks.patch}:

\end{itemize}

Next two patches provide a multi-block (~\url{mballoc}) allocator to the ldiskfs
filesystem.  They are also used in the Linux ext4 filesystem. Just to clarify
at this point, ~\url{extent} and~\url{mballoc} allocations are orthogonal to
each other. When allocating an~\url{extent} a contiguous set of blocks is
allocated. Let's assume the first block in this~\url{extent} allocation starts
from $ x $ and the last block is $ x + y $. The next~\url{extent} to be
allocated for the same file might or might not start from $ x + y + 1 $, but in
any case, this new~\url{extent} will be allocated to a new set of contiguous
blocks. This way, while addressing these two perhaps disjoint sets of
blocks on disk, we only need to provide their~\url{extent} information instead
of addresing them block by block. ~\url{mballoc}, on the other hand, is a way
to allocate the requested amount of (hopefully contiguous) blocks with one call.


\begin{itemize}

\item \url{ext3-mballoc3-core.patch}:

\item \url{ext3-mballoc3-sles10.patch}: 

\item \url{ext3-nlinks-2.6.9.patch}: allows more than 32,000 subdirectories to
be created in a given directory. This is achieved by setting~\url{i_nlink} to 1
if overflowed (if number of subdirectories is greater than 32,000, addressed by
a 15 bit~\url{i_nlink} counter) (also used in the Linux ext4 filesystem).

\item \url{ext3-ialloc-2.6.patch}: changes the inode allocation policy for OSTs
to avoid full groups (i.e., to skip full groups). There is a bitmap at the
beginning of each inode group to show which group is fully used, which group is
partially used, and which group is free. The superblock contains information
about how many inodes are free in different inode groups. This patch gives
information about an inode group, whether there are any free inodes in that
group or not, without making a full inode group scan. 

\item \url{ext3-disable-write-bar-by-default-2.6-sles10.patch}: turns off write
barriers in a journal which cause cache flushes. 

\item \url{ext3-uninit-2.6-sles10.patch}: provides identifying uninitialized
block groups for increasing the e2fsck performance (also used in the Linux ext4
filesystem). 

\item \url{ext3-nanosecond-2.6-sles10.patch}: provides nanosecond resolution
timestamps for inodes (also used in the Linux ext4 filesystem).

\item \url{ext3-inode-version-2.6-sles10.patch}: provides inode version on disk
for version based recovery (also used in the Linux ext4 filesystem).

\item \url{ext3-mmp-2.6-sles10.patch}: provides multi-mount protection to avoid
double mount in Linux High-Avaliabilty (HA) environment.

\item \url{ext3-fiemap-2.6-sles10.patch}: provides file extent mapping API
(efficient fragmentation reporting) (also used in the Linux ext4 filesystem).

\item \url{ext3-statfs-2.6-sles10.patch}: provides~\url{statfs} speedup by just
gathering info from superblock without diving deeper (also used in Linux ext4
filesystem).

\item \url{ext3-block-bitmap-validation-2.6-sles10.patch}: verifies bitmaps on
disk are sane to avoid cascading corruption (also used in the Linux ext4 filesystem).

\item \url{ext3-get-raid-stripe-from-sb.patch}: stores RAID layout in ext3
superblock to optimize allocation (used by allocator and the information is
 written by mkfs) (also used in the Linux ext4 filesystem).

\end{itemize}
