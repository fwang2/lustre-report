\section{Future Work}

Lustre was initiated and funded, almost a decade ago, by the U.S. Department of
Energy (DOE) Office of Science and National Nuclear Security Administration
(NNSA) Laboratories to address the need for an open source, highly scalable,
high-performance parallel filesystem on then-present and future supercomputing
platforms. Throughout the last decade, while satisfying the scalability and
performance requirements of the various supercomputing platforms deployed not
only by DOE Laboratories but also by other domestic and international industry
and research institutes, Lustre has become increasingly large and complex.
This report only scratches the surface of the current 1.6 version of the Lustre
source code base. Because Lustre is a moving target, keeping documentation up
to date to address day-to-day user problems and to help meet future
requirements will require a substantial effort. Limited by time and resources
the authors did not include the following topics that merit documentation: 


\begin{itemize}

\item This documentation is based on Lustre code base \url{b1.6} as the
reference implementation. However, significant changes occurred on the \url{b1.8}
branch and the upcoming 2.0 release. Any future effort should take this into
consideration.

\item Failure recovery has been a constant theme for many bug fixes and new
Lustre feature development. It also has a direct bearing on providing insight on the 
Lustre diagnosis we do on a daily basis. Though we touch on it in
this report, it would be desirable to provide a holistic view on the subject
from the Lustre kernel support perspective.


\item Lustre scalability is another topic in which we have a vested interest.
A centralized discussion on its kernel support, status, limits, and recent bug
fixes and feature enhancement would be highly beneficial.

\item Quota support was also left out of this report. A detailed analysis of
this subject in future updates will be benefical for both the Lustre developers
and the user community.

\item Security (e.g., Kerberos, in terms of computer network authentication)
will soon be a default requirement for Lustre. Coverage of this topic in future
updates would provide  a more complete picture of Lustre. 

\end{itemize}

The authors have shared their insights and understanding of Lustre as it stands
today. The documentation on Lustre filesystem internals may never be complete
because of the ever-changing nature of the Lustre source code base. It is the
authors' hope that the Lustre community will make a collective effort to
continue this work.
